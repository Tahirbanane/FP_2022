\section{Diskussion}
\label{sec:Diskussion}

Auffällig bei manchen Messungen war, dass die Ergebnisse nicht immer den erwarteten Werten entsprachen beziehungsweise nicht immer reproduzierbar waren. So ist aufgefallen,
dass bei der Messung der Winkel bei ~2300Hz und einem Zwischenring, manchmal keine Maxima gemessen wurden oder diese sehr viel niedriger waren als erwartet. 

\noindent
Wiederholten messen brachte jedoch manchmal die erwartete Höhe des Maximas. Dies kann an einem Wackelkontakt oder an einem Softwarefehler in der Messaperatur liegen.
Mithilfe dieser Begründung können eventuell auch die Abweichungen der Messwerte von der Theorie mit erklärt werden, welche in \autoref{fig:4} zusehen sind.

\noindent
Die Bänder und die Bandlücken verhielten sich wie erwartet aus der Theorie.

\noindent
Der Versuch hat eindrucksvoll leicht einen Einblick in die komplexe Welt der Festkörperphysik und Quantenphysik bereitet. Die Systeme verhielten sich alle wie 
theoretisch erwartet. Somit konnten Schlüsse auf die Atomorbitale, die Aufspaltung der Resonanzen bei Symmetriebrechungen und deren Verhalten eines Wasserstoffatoms
analysiert und die Quantenzahlen l und m bestimmt werden.


