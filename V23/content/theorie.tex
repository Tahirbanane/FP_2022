\section{Zielsetzung}
Ziel dieses Experiments ist es, quantenmechanische Modelle mit akustischen Systemen zu untersuchen und zu simulieren.
Dazu werden verschiedene Hohlraumresonatoren verwendet.


\section{Theoretische Grundlagen}

\noindent
Schall ist eine Longitudinal-Welle die sich in einem Gas wie Luft ausbreitet.
In diesem Versuch werden akustische Systeme aus Resonatoren aufgebaut.
Schallwellen können mithilfe der Helmholtzgleichung 

\begin{equation*}
    \Delta P(\vec{r},t) = \frac{1}{c^2} \frac{\partial^2 P(\vec{r},t)}{\partial t^2}
\end{equation*}

\noindent
als eine Verteilung der Druckamplitude $P(\vec{r},t)$ beschrieben werden.
Dabei ist $c$ die Ausbreitungsgeschindigkeit der Welle.

\noindent
Mit dem Separationsansatz $P(\vec{r},t) = p(\vec{r}) \cdot \cos(\omega t)$ ergibt sich eine stationäre Differentialgleichung für den Druck.

\begin{equation}
    \Delta p(\vec{r}) = \frac{\omega^2}{c^2} p(\vec{r})
    \label{eqn:helmholtz}
\end{equation}

\noindent
Das quantenmechanische Analogon ist die Schrödingergleichung

\begin{equation*}
    -\frac{\hbar^2}{2m} \Delta \Psi(\vec{r},t) + V(\vec{r}) \Psi(\vec{r},t) = i\hbar \frac{\partial}{\partial t} \Psi(\vec{r},t)
\end{equation*}

\noindent
deren Betragsquadrat die Wahrscheinlichkeitsdichte des Elektrons ist.

\noindent
Mit dem Separationsansatz $\Psi(\vec{r},t) = \psi(\vec{r}) \cdot e^{-i \omega t}$ ergibt sich die stationäre Schrödingergleichung

\begin{equation}
    -\frac{\hbar^2}{2m} \Delta \psi(\vec{r}) + V(\vec{r}) \psi(\vec{r}) = E \psi(\vec{r}).
    \label{eqn:schroedinger}
\end{equation}

\subsection{Der eindimensionale Festkörper}
\noindent
Für eine stehende Welle in einem unendlich hohen Potentialtopf der Länge $L$ müssen die Randbedingungen $ \psi(0) = \psi(L) = 0$ erfüllt sein.
Mithilfe der stationären Schrödingergleichung \ref{eqn:schroedinger} ergibt sich eine ebene Welle  $\psi(x) = A \sin(kx)$ mit der Wellenzahl $k$. Unter Berücksichtigung der Randbedingungen lautet $k$

\begin{equation}
  k = \frac{n \pi }{L} \qquad \text{mit} \qquad n \in \mathbb{N}
\end{equation}

\noindent
Damit sich eine stehende Welle in einem Hohlraumresonator der Länge $L$ ausbildet, muss gelten:

\begin{equation}
L = \frac{n\lambda}{2}
\end{equation}

\noindent
Über die Wellenzahl $k=\frac{2\pi}{\lambda}$ kann die Bedingung umgeschrieben werden in 

\begin{equation}
  k = \frac{n \pi }{L} 
\end{equation}

\noindent
und wir erhalten die gleiche Bedingung wie für den Potentialtopf.


\noindent
Durch Aneinanderreihung mehrerer Aluminiumzylinder mit Irisblenden zwischen diesen, kann die Kopplung mehrerer Potentialtöpfe simuliert werden. Dies kann als Analog zu einem Übergang von einem Atom, über ein Molekül zu einem 1-dim. Festkörper verstanden werden.

\noindent
Freie Elektronen haben die parabolische Dispersionsrelation

\begin{equation*}
  E = \frac{\hbar^2 k^2}{2m}.
\end{equation*}

\noindent
Die freien Elektronen werden an den periodisch angeordneten Kernen des Festkörpers mit Abstand $d$ gestreut, wobei die Bragg Bedingung

\begin{equation*}
  n \lambda = 2d 
\end{equation*}

\noindent
erfüllt sein muss. Somit gilt für die Kreiszahl

\begin{equation*}
k = \frac{n \pi}{d} 
\end{equation*}

\noindent
Im akustischen Analogon simulieren durch Blenden verbundene Hohlraumzylinder die Einheitszellen des 1-dim. Festkörpers, wobei die Blenden die Streuzentren darstellen.


\subsection{Das Wasserstoffatom}

\noindent
Das Wasserstoffatom unterliegt der Kugelsymmetrie des Coulombpotentials des geladenen Kerns $V(r) = -\frac{e^2}{4 \pi \epsilon_0 r}$.
In Kugelkoordinaten kann die stationäre Schrödingergleichung \ref{eqn:schroedinger} mit dem Separationsansatz  $\psi(r,\varphi,\theta) = Y^m_l(\varphi,\theta) \cdot R_{n,l}(r)$
in einen winkelabhängigen Teil

\begin{equation*}
    -\left[\frac{1}{\sin \theta} \, \frac{\partial}{\partial \theta} \left(\sin \theta \, \frac{\partial}{\partial \theta}\right) + \frac{1}{\sin^2 \theta} \, \frac{\partial^2}{\partial \phi^2}\right] Y^m_l(\varphi,\theta) = l(l+1) \, Y^m_l(\varphi,\theta)
\end{equation*}

\noindent
und einen radialen Teil aufgeteilt werden, wobei der radialabhängige Teil nicht relevant ist, da er im Kugelresonator nicht realisiert werden kann.
Der winkelabhängige Teil enthält die Kugelflächenfunktion

\begin{equation}
    Y^m_l(\varphi,\theta) = \frac{1}{\sqrt{2\pi}} \; \sqrt{\frac{2l+1}{2} \; \frac{(l-m)!}{(l+m)!}} \; P_{lm}\left(\cos \theta\right) e^{i m \varphi}
\end{equation}

\noindent
mit den Legendre-Polynomen $P_{lm}$.
Die auftretenden Quantenzahlen $n$, $l$ und $m$ sind dabei wie folgt definiert:

\begin{align*}
\text{Hauptquantenzahl  }    n &= 1, \, 2, \, \ldots  \\
\text{Drehimpulsquantenzahl  }    l &= 0, \, 1, \, \ldots, \, n-1 \\
\text{magnetische Quantenzahl  }    m &= -l, \, -l+1, \, \ldots, \, l-1, \, l
\end{align*}

\noindent
Die Hauptquantenzahl $n$ beeinflusst die Energieeigenwerte $E = -\frac{E_{\text{ryd}}}{n^2}$.
Die Energieeigenwerte sind jedoch nicht von $l$ oder $m$ abhängig und damit in diesen entartet. 
Die Entartung in $l$ ist das Resultat des 1/r-Potentials, analog zur Erhaltung des Lenz-Runge-Vektors im Gravitationspotential, wohingegen die Entartung in $m$ aus der sphärischen Symmetrie des Wasserstoffproblems folgt.

\noindent
Das akustische Analogon ist der Kugelresonantor, wobei die Helmholtzgleichung \ref{eqn:helmholtz} mit dem Separationsansatz  $p(r,\varphi,\theta) = Y^m_l(\varphi,\theta) \cdot F_{n,l}(r)$ in radialen und winkelabhängigen Teil aufgeteilt wird.
Der winkelabhängige Teil wird durch die Kugelflächenfunktionen gelöst, der radiale Teil wird in diesem Versuch nicht beobachtet, da die $r$-Abhängigkeit die im Wasserstoffatom durch das Coulomb-Potential verursacht wird, im Kugelresonator nicht realisiert werden kann.
Durch einsetzen eines Zwischenrings im Kugelresonator wird ein Magnetfeld im Wasserstoffatom simuliert, welches die Entartung in $m$ aufhebt.
Die Entartung in $l$ ist mangels Coulombpotential nicht realisiert und zu verschiedenen $l$-Werten gehören verschiedene Resonanzfrequenzen.

\subsection{Das Wasserstoffmolekül}

\noindent
Das $\text{H}^+_2$-Molekül besteht aus einem Elektron welches sich im Coulomb-Potential zweier Kerne aufhält.
Die sich dabei überlappenden Atomorbitale können dabei bindend oder antibindend sein.
Bindende Überlappung liegt vor wenn die Wellenfunktionen das gleiche Vorzeichen haben, also symmetrisch (gerade) zueinander stehen.
Antibindende Überlappung liegt bei unterschiedlichen Vorzeichen vor, wenn die Wellenfunktionen antisymmetrisch (ungerade) sind.
Somit gibt es einen $2\sigma_{u/g}$-Zustand, der aus der Bindung zweier $l=1$, $m=0$-Zustände zusammengesetzt ist. Ebenso gibt es einen $1\pi_{u/g}$-Zustand, der aus zwei gebundenen $l$=1, $m$=1-Zuständen besteht.

\noindent
Das Wasserstoffatom wird in diesem Versuch durch zwei gekoppelte Kugelresonatoren simuliert, wobei Blenden eingesetzt werden können die im Analogon die Kopplungsstärke oder den Abstand der Wasserstoffkerne realisieren.
Zwischen bindenden und antibindenden Zuständen kann durch eine Phasenverschiebung von 180° unterschieden werden.

 