\section{Zielsetzung}
Ziel des Versuches ist es, die Dipolrelaxation von Ionenkristallen zu untersuchen.
Dafür soll die Relaxationszeit der Dipole sowie die Aktivierungsenergie ermittelt werden.

\section{Theoretische Grundlagen}
\subsection{Dipole in Ionenkristallen}
Ein Ionenkristall ist eine Gitterstruktur, in der Kationen und Anionen alternierend angeordnet sind und über Ionenbindung zusammengehalten werden.
In diesem Versuch wird durch Dotierung mit doppelt positiv geladenen Strontium Atomen ein Ladungsungleichgewicht erzeugt.
Durch die geforderte Ladungsneutralität wird ein einfach positiv geladenes Kation nach außen verdrängt und es ensteht eine Leerstelle.
Zwischen dem Strontium Atom und der Leerstelle liegt nun eine Ladungsdifferenz vor und ein Dipolmoment liegt an, welches zur Leerstelle hin ausgerichtet ist.

\noindent
Da bei Raumtemperatur die Zustandswahrscheinlichkeiten näherungsweise durch die Boltzmann-Statistik gegeben sind, ist das Gesamtdipolmoment null.
Eine Umorientierung der Dipolmomente kann durch Zuführung thermischer Energie hervorgerufen werden, wenn dabei die nötige Aktivierungsenergie $W$ aufgebracht wird.
Die Zeit die ein Dipolmoment für eine Umorientierung benötigt, ist durch 

\begin{equation}
    \label{eqn:tau}
    \tau(T) = \tau_0 e^{\frac{W}{k_B T}}
\end{equation}

\noindent
gegeben, wobei $\tau_0$ die charakteristische Relaxationszeit für $\tau(T \to \infty)$.
Befindet sich der Ionenkristall in einen elektrischen Feld, richten sich die Dipolmomente entlang der Feldlinien aus.
Wird ein erwärmter Kristall in einem elektrischen Feld abgekühlt, ändert sich die Ausrichtung der Dipolmomente auch nach abschalten des elektrischen Feldes nicht, da sich die Leerstellen bei niedrigen Temperaturen nicht mehr verschieben.
Wenn der Kristall wieder langsam erwärmt wird, kommt es zur Relaxation und die Dipolmomente richten sich zurück in ihren Anfangszustand aus.
Geschieht dies in einem unaufgeladenen Plattenkondensator, kann durch Induktion ein Polarisationsstrom gemessen werden.
In den folgenden Kapiteln soll der Polarisationsstrom hergeleitet werden.

\subsection{Herleitung des Polarisationsstroms über die Stromdichte}
Der Polarisationsstrom ist im Allgemeinen gegeben durch

\begin{equation}
    \label{eqn:j}
    j(T) = -\frac{dP(t)}{dt} = \frac{P(t)}{\tau(T)}
\end{equation}

\noindent
mit der Polarisation 

\begin{equation}
    \label{eqn:p}
    P(t) = P_0 \text{exp} \Bigl( - \frac{t}{\tau(T)} \Bigr).
\end{equation}

\noindent
Wird Gleichung \ref{eqn:p} in Gleichung \ref{eqn:j} eingesetzt und der Exponent in Integralschreibweise ausgedrückt, ergibt sich

\begin{equation}
    j(T) = \frac{P_0}{\tau} \text{exp} \Bigl( - \int_0^t \frac{\text{d}t}{\tau(T)} \Bigr),    
\end{equation} 

und mit der Heizrate $b := \frac{dT}{dt}$:

\begin{equation}
    j(T) = \frac{P_0}{\tau} \text{exp} \Biggl( - \int_{T_0}^T exp \Bigl( -\frac{W}{k_B T'} \Bigr) \text{d}T' \Biggr).
\end{equation}
\noindent
ausdrücken.


\subsection{Herleitung des Polarisationsstroms über  den Polarisationsansatz}
Für hohe Temperaturen $p E << k_B T $ ist die genäherte Polarisation durch 

\begin{equation}
    P = \frac{N p^2 E}{3 k_B T} 
\end{equation}

\noindent
gegeben. Dabei ist $N$ die Dichte der Dipole pro Volumen und $P(T_p)$ die Polarisation der ausgerichteten Dipole, die mit der Stromdichte

\begin{equation}
    j(T) = P(T_p) p \frac{\text{d}N}{\text{d}t} .
\end{equation}

\noindent
Mit dem Zusammenhang $\frac{\text{d}N}{\text{d}t} = -\frac{N}{\tau{T}}$ ergibt sich
\begin{equation}
    N = N_p \text{exp} \Biggl( -\frac{1}{b} \int_{T_0}^{T} \frac{\text{d}t'}{\tau(T')} \Biggr),  
\end{equation}

\noindent
wobei $N_p$ die Zahl der bestehenden ausgerichteten Dipole pro Volumen zum Beginn des Heizprozesses beschreibt.
Schließlich lässt sich der Polarisationsstrom durch

\begin{equation}
    j(T) = \frac{p^2 E N_p}{3 k_B T_p \tau_0} \text{exp} \Biggl(-\frac{1}{b} \int_{T_0}^{T} \text{exp} \Bigl( - \frac{W}{k_B T'} \Bigr) \text{d}T' \Biggr)\cdot 
    \text{exp} \Bigl( - \frac{W}{k_B T'} \Bigr)
\end{equation}

\noindent
ausdrücken, wobei für tiefe Temperaturen die Näherung

\begin{equation}
    j(T) \approx \frac{p^2 E N_p}{3 k_B T_p \tau_0} \text{exp} \Bigl( - \frac{W}{k_B T'} \Bigr)
\end{equation}

\noindent
gilt.

\subsection{Die Aktivierungsenergie}
Die Änderung der Polarisation führt  zu einem einem Strom um den Querschnitt $F$ und es gilt

\begin{equation}
    \label{eqn:kp}
\frac{I(t)}{F} = \frac{\text{d}P}{\text{d}t}.
\end{equation}

\noindent
Durch Umformumg und Integration lässt sich Gleichung \ref{eqn:kp} schreiben als

\begin{equation}
    \int_{t(T)}^{\infty} I(t) \text{d}t = - F P(t).
    \end{equation}

\noindent
Mit Gleichungen \ref{eqn:j} und \ref{eqn:tau} folgt für die Aktivierungsenergie

\begin{equation}
    \label{eqn:idkwhat}
W= k_B T \:\text{ln} \left( \frac{\int_{T}^{\infty} I(T') \text{d} T'}{I(T) \tau_0 b}\right).
\end{equation}
