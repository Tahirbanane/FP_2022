\section{Diskussion}
\label{sec:Diskussion}

Wenn die Heizraten, welche mit unterschiedlichen Methoden berechnet wurden, mit einander verglichen werden, dann fällt auf, dass diese etwas von einander abweichen. Diese lässt sich 
nach 
\begin{equation*}
    \Delta W_{\%} = \frac{W_1-W_2}{W_1}
\end{equation*}
\noindent
berechnen, wobei man $W_1$ und $W_2$ die zu untersuchenden Zahlen sind und $\Delta W_{\%}$ die Abweichung von $W_1$ zu $W_2$ angibt.

\begin{align*}
    \Delta W_{1,\%} &=  9,49\%   \\
    \Delta W_{2,\%} &=  6,84\%
  \end{align*}

Dies kann zum einen daran liegen das ein paar Werte nicht ausgewertet werden konnten, da Aufgrund der Approximation des Untergrundes negative Zahlenwerte, sowie nullen, enstanden sind, welche
im Logarithmus nicht zum einem finiten Zahlenwert führen bzw. nicht definiert sind. Daher fehlen diese auch in den Plots. Besonders stark viel dies bei der 2. Messung ins Gewicht.
Ein anderer Grund an der Wahl der Intervalle die Untersucht werden.

\noindent
Zudem kann der Mittelwert der Aktivierungsenergie aus den beiden Methoden gemittelt werden um dies mit dem Theoriewert vergleichen zu können. Dieser beträgt $W_\text{theo} = 0,66$ eV \cite{muccillo}.

\begin{align*}
    W_\text{gemittelt,1} &= 0,725  \si{\electronvolt}\\
    W_\text{gemittelt,2} &= 0,7955 \si{\electronvolt}\\
    \Delta W_{1,\%} &=  9,85\%   \\
    \Delta W_{2,\%} &=  20,53\%
  \end{align*}

\noindent
Das selbe kann auch für die Relaxationszeit gemacht werden. Der Theoriewert für die Relaxationszeit ist bei $\tau_\text{theo} = 4 \, \cdot 10^{-14}$s. 

\begin{align*}
    \tau_\text{gemittelt,1} &= 1,02 \cdot 10^{-13} \si{\second}\\
    \tau_\text{gemittelt,2} &= 0,90035 \cdot 10^{-14} \si{\second}\\
    \Delta \tau_{1,\%} &=  155\%   \\
    \Delta \tau_{2,\%} &=  77,49\% \\
\end{align*}

\noindent
Diese starken Abweichungen können aus unterschiedlichen Gründen zustandekommen. Es gab hin und wieder bei dem Ablesen der Werte Probleme, da leichte Erschütterungen schon die Naddel
wackeln liesen, genau wie Bewegungen in der Nähe des pico-Ampermeters. Zudem war die Heizrate nicht ganz konstant und musste immer wieder händisch nachgeregelt werden.

\noindent
Die starken Abweichungen in der zweiten Methode werden aus den selben Gründen, wie bereits, oben erwähnt zustandekommen. Zudem sollte erwähnt werden, dass sich bei der zweiten Methode
die Abweichung zwischen der Messung 1 und Messung 2 um den Faktor $10^{3}$ unterscheiden.
