\section{Ziel}
Ziel des Versuches war es die Zusammensetzung eines 3x3-Würfels in einer Ebene zu bestimmen, wobei die einzelnen Teilwürfel aus unterschiedlichen Metallen besteht. 

\section{Theorie}
\subsection{Tomographie}
Die Tomographie ist ein Bild-gebendes Verfahren, welches viel Anwendung in der heutigen Medizin findet. Besonders die so genannte Computertomographie, kurz CT, ist weit 
verbreitet.
Durch dieses Verfahren werden Querschnitte erzeugt und durch die Untersuchung mehrerer Schichten kann so ein 3 Dimesnionales Bild generiert werden.

\noindent
Im Allgemeinen wird für die Tomographie $\gamma$-Strahlung benutzt. Durch die unterschiedlichen Absorptionskoeffizienten und durch die Bestrahlung des Targets aus 
verschiedenen Winkeln kann ein Bild erzeugt werden.

\subsection{Wechselwirkung von Materie mit Gamma-Strahlung}
$\gamma$-Strahlung wechselt wirkt hauptsächlich in 3 Art und Weisen mit Materie. Diese sind der Photoeffekt, die Compton-Streuung, sowie die Paarerzeugung.
Beim Photoeffekt wird ein Photon vollständig von einem gebundenen Elektron absorbiert, sodass dieses aus seiner Bindung herausgelöst wird. 
Der Wirkungsquerschnitt pro Dichte 

daher dominiert im Allgemeinen bei einer Energie <100keV 




\subsection{Wechselwirkung von Gamma-Strahlung mit Materie}
Dieses Gesetz lässt sich gut auf $\gamma$-Strahlung anwenden, die in guter Näherung
nur eine Wehcselwirkung pro $\gamma$-Quant durchführt. Quelle der $\gamma$-Strahlung ist
hierbei das Abfallen angeregter Kernzustände auf energetisch stabilere, energieärmere
Zustände. Die dabei verlorene Energie wird großteils in Form eines $\gamma$-Quants frei.
Da die möglichen Energieniveaus eines gebundenen Elektrons diskret sind, ist auch das
mögliche Spektrum der $\gamma$-Strahlung diskret. Die möglichen Wechselwirkungen mit Materie
lassen sich dabei in Annihlationsprozesse sowie elastische- und inelastische Streuung
einteilen. Bei den $\gamma$-Energien von
\SI{10}{\kilo\electronvolt} bis \SI{10}{\mega\electronvolt} die in diesem Experiment
erreicht werden, treten vor allem die folgenden drei Arten von Wechselwirkungen auf:
\begin{enumerate}
  \item \textbf{Photo-Effekt}: Beim (inneren) Photo-Effekt, einem Annihlationseffekt,
  schlägt das einfallende $\gamma$-Quant ein Elektron aus der Hülle eines Teilchens.
  Das $\gamma$-Quant wird dabei vernichtet, das Elektron erhält eine kinetische
  Energie, die von der Energie des einfallenden $\gamma$-Teilchens ($E_\gamma = \symup{h}\nu$)
  sowie der Bindungsenergie $E_{\symup{B}}$ des Elektrons abhängig und durch
  \begin{equation}
    E_{\symup{e}} = E_\gamma - E_{\symup{B}}
    \label{eqn:2}
  \end{equation}
  gegeben ist. Die Bindungsenergie des Elektrons bietet daher eine untere Schranke
  für die Energien, bei denen der Photo-Effekt eintritt. Die Impulserhaltung sorgt
  weiter dafür, dass der Photo-Effekt in den inneren, stärker gebunden Elektronenschalen
  wahrscheinlicher ist. Außerdem steigt die Wahrscheinlichkeit aus dem selben Grund
  für schwere Atome, da die Bindungsenergie hier stärker ist.
  \item \textbf{Compton-Effekt}: Als Compton-Effekt bezeichnet man die inelastische Streuung
  von $\gamma$-Quanten an freien Elektronen. Bei hohen Energien kommen dafür auch die äußeren
  Hüllenelektronen in Frage. Das $\gamma$-Quant wird dabei nicht vernichtete, es verliert einen
  Teil seiner Energie und beschleunigt dabei das getroffene Elektron. Beide Teilchen bewegen
  sich dabei weiter, ändern ihre Richtung jedoch (sie werden "gestreut"). Dadurch nimmt
  die Intensität des Teilchenstrahls ab. Der dabei zu betrachtene Wirkungsquerschnitt bestimmt
  sich jedoch nicht nach dem weiter oben genannten Zusammenhang, sondern durch:
  \begin{equation}
    \sigma_{\symup{com}} = 2 \pi r_{\symup{e}}^2 \left\{ \frac{1+\epsilon}{\epsilon^2} \left[ \frac{2(1+\epsilon)}
    {1+2\epsilon} - \frac{1}{\epsilon} \ln(1+2\epsilon) \right] + \frac{1}{2\epsilon}
    \ln(1+2\epsilon) - \frac{1+3\epsilon}{(1+2\epsilon)^2} \right\}
    \label{eqn:5}
  \end{equation}
  mit $\epsilon = \symup{E}_\gamma / m_0c^2$. Mit \eqref{eqn:5}
  kann man nun aus der Massenzahl $z$, der Avogadro-Konstante $N_{\symup{A}}$,der Dichte $\rho$ und
  der Molmasse $M$ durch:
  \begin{equation}
    \mu_{\symup{com}} = \frac{z N_{\symup{A}} \rho}{M} \cdot \sigma_{\symup{com}}(\epsilon)
    \label{eqn:4}
  \end{equation}
  den Absorbtionskoeffizienten $\mu$ bestimmen. Für $\gamma$-Quanten, deren
  Energie klein im Vergleich zur Ruheenergie des getroffenen Elektrons ist. $\sigma$ bestimmt sich
  dann näherungsweise als
  \begin{equation}
    \sigma = \frac{8}{3} \pi \symup{r}_{\symup{e}}^2
  \end{equation}
  mit dem klassischen Elektronenradius $\symup{r}_{\symup{e}}$.
  \item \textbf{Paarbildung}: Für sehr große $\gamma$-Energien oberhalb der doppelten
  Ruhemasse der Elektronen, wird das $\gamma$-Quant unter Bildung eines Elektron-Positron-Paares
  annihliert. Der Betrag $2 \symup{m}_0 \symup{c}^2$ ist hierbei als untere Schranke
  nicht ausreichend, da ein weiterer Stoßpartner notwendig ist, der einen Teil des Impulses
  des $\gamma$-Quants aufnimmt, damit der Vorgang der Impulserhaltung genügt.
\end{enumerate}
Die oben genannten Effekte treten beim Durchgang eines $\gamma$-Strahls durch ein
Absorbermaterial zusammen auf und überlagern sich daher, es kann jedoch in bestimmten
Energiebereichen eine Dominanz bestimmter Effekte beobachtet werden. Der Verlauf einer solchen
Kurve ist in Abbildung \ref{abb:1} dargestellt.


Quelle: https://github.com/FeGeyer/praktikum/blob/master/4_Semester/V704/body.tex


\subsection{Fehlerbestimmung}