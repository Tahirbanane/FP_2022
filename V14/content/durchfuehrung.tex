\section{Aufbau und Durchführung}

\subsection{Aufbau}
Der Aufbau des Versuches ist relativ simpel. Auf der einen Seite befindet sich die Strahlungsquelle, in unserem
Fall Cs-137, welches von großes Bleiböcken abgeschirmt, sodass ein schmaler Strahlengang entsteht. Gegenüber 
von der Quelle sitzt ein NaJ-Detektor. Mit diesem wird die $\gamma$-Strahlung detektiert. 

\noindent
Unterhalb des Strahlengangs befindet sich eine Platte, auf welche sich Proben montieren lassen. Diese lässt 
sich verschieben und drehen. Im Versuch wurden auf dieser mehrere Würfel platziert. Einer war leer und wurde genutzt um eine Untergrundsmessung durchzuführen und $I_0$ zu messen,
2 waren vollständig homogen mit einem umbekannten Metall gefüllt und 1 Würfel bestand aus kleineren Würfeln aus unterschiedlichen Stoffen, welche den großen Würfel 
im Format 3x3x3 füllten. 

\noindent
Da die Platte nicht in der Höhe verstellbar war wurde lediglich die 2. Schicht des Würfels gemessen, sodass nur eine Bestimmung von 9 kleineren Würfeln möglich war. Der Aufbau ist
in der \autoref{fig:aufbau} dargestellt.

\begin{figure}
    \centering
    \includegraphics[width=0.75\textwidth]{aufbau.png}
    \caption{Versuchsaufbau \cite{V14}}
    \label{fig:aufbau}
  \end{figure}

\noindent
Als mögliche Stoffe kommen nur Aluminium, Blei, Eisen, Messing ($Cu_{0.63}Zn_{0.37}$) und Delrim (POM) infrage \cite{V14}. 

\noindent
Als zusätzliche Aufbauten, neben der $\gamma$-Quelle und dem Würfel, wird ein NaI-Detektor genutzt um die Ausgangsstrahlung zu messen, wie ein Multichannelanalyzer.
Ein Multichannelanalyzer (kurz MCA) kann genutzt werden um elektrische Impulse mit unterscheidlichen Aplituden ihrer Häufigkeit nach zu sotieren. 

\subsection{Szintillationsdetektor}
Mithilfe eines Szintillationsdetektors kann die Energie und Intensität von ionisierender Strahlung bestimmt werden. Dieser kann aus organischen, aber auch aus anorganischen Stoffen
bestehen. Der im Versuchsaufbau verwendete Szintillationsdetektor nutzt einen anorganischen Natriumiodidkristall, welcher, wenn eine Energie größer als die Anregungsenergie 
dessen in ihn eindringt, sogenannte "Lichtblitze" erzeugt, die durch einen Bandwechsel von angeregten Elektronen entstehen. Die freigesetzten Photonen werden mithilfe eines 
Photomultipliers detektiert. Der Strom der in dem Photomultiplier entsteht ist propotional zu der Energie des $\gamma$-Quants, welches für die Erzeugung des detektierten Photons
durch den Photomultiplier verantwortlich ist. 


\subsection{Durchführung}
Zur Hilfe mit der Messung wird das Programm MAESTRO genutzt.

\noindent
Zuerst wird ein Spektrum der verwendeten Quelle genutzt. Gemessen wird solange bis mindestens am Piek 1112 Impulse gemessen wurden. Diese Zahl kommt Zustande da ein Fehler < 0,03
gewünscht ist und mithilfe des Zusammenhangs

\begin{equation}
    \frac{\sqrt{N}}{N} = 0,03 \leftrightarrow N = \frac{1}{0,03^2} \approx 1112
    \label{eqn:n}
  \end{equation}
  
kommt dieser Zustande. Das Programm zeigt dies an, sodass einfach die Messung auf ein USB Stick expotiert werden können.

\noindent
Bei dem leeren Würfel, dem 2 Würfel und dem 3x3x3 Würfel wurden alle Strahlengänge nach der \autoref{fig:projektion} gemessen. Beim 3. Würfel wurden lediglich die Strahlengänge
1,2,3,10,11,12 gemessen. 

\noindent
Da das Programm automatisch nach 300s aufhört musste bei dem 3. Würfel der Strahlengang 11 und beim 3x3 Würfel der Strahlengang 8 und 11 2 mal gemessen und später in der
Auswertung addiert werden, um die Messunsicherheiten gering zu halten.












