\section{Aufbau und Durchführung}
\subsection{Aufbau}
Der Aufbau des Versuches ist relativ simpel. Auf der einen Seite befindet sich die Strahlungsquelle, in unserem
Fall Cs-137, welches von großes Bleiböcken abgeschirmt, sodass ein schmaler Strahlengang entsteht. Gegenüber 
von der Quelle sitzt ein NaJ-Detektor. Mit diesem wird die $\gamma$-Strahlung detektiert. 

\noindent
Unterhalb des Strahlengangs befindet sich eine Platte, auf welche sich Proben montieren lassen. Diese lässt 
sich verschieben und drehen. Im Versuch wurden auf dieser mehrere Würfel platziert.

\subsubsection{Szintillationsdetektor}


\subsubsection{Multichannelanalyzer}



\subsection{Durchführung}



