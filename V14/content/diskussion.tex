\section{Diskussion}
\label{sec:Diskussion}

\noindent
Nun werden die experimentell bestimmten Absorptionskoeffizienten mit den Literaturwerten aus Tabelle \ref{tab:lit} verglichen.


\begin{table}
    \centering
    \sisetup{table-format=2.1}
    \begin{tabular}{c c c c c}
    \toprule
    Eisen & Aluminium & Blei & Messing & Delrin\\
     \midrule 
    0.578 & 0.202 & 1.245 & 0.62 & 0.118\\
\bottomrule
\end{tabular}
\caption{Literaturwerte der Absorptionskoeffizienten \cite{chemie} \cite{delrin}.}
\label{tab:lit}
\end{table}
    
\noindent
Dem homogenen Würfel 2 mit dem bestimmten Absorptionskoeffizient $\mu_2 = 0.1029 \, \frac{1}{cm}$ ist mit einer relativen Abweichung von 12.79\% das Material Delrin zuzuordnen.

\noindent
Würfel 3 kann mit dem bestimmten Wert $\mu_3 = 0.9827 \, \frac{1}{cm}$ mit einer Abweichung von 21.07\% das Material Blei zugeordnet werden.

\noindent
Die bestimmten Absorptionskoeffizienten von Würfel 4 befinden sich in Tabelle \ref{tab:w4}.
Im folgenden werden erneut die negativen Werte nicht berücksichtigt, da sie physikalisch nicht sinnvoll sind.
In Tabelle \ref{tab:lit} befinden sich die zugeordneten Materialien und relative Abweichungen der einzelnen Würfel.

\begin{table}
    \centering
    \sisetup{table-format=2.1}
    \begin{tabular}{c c c c c}
    \toprule
    Einzelwürfel & Material & rel. Abw. / \% \\
     \midrule 
    1 & Aluminium & 9.65\\
    2 & Messing & 2.31 \\
    3 & Aluminium & 81.34\\
    4 &  & \\
    5 & Blei & 2.66 \\
    6 &  & \\
    7 & Delrin &21.36\\
    8 & Blei & 21.49\\
    9 &  &\\
\bottomrule
\end{tabular}
\caption{Materialien der Elementarwürfel in Würfel 4.}
\label{tab:lit}
\end{table}

\noindent
Aufgrund der teilweise hohen Abweichungen und der Vernachlässigung der negativen Absorptionskoeffizienten aus Tabelle \ref{tab:muj} wird die Messung von Würfel 4 als nicht erfolgreich angesehen.

\noindent
Mögliche Fehlerquellen dafür sind systematische Fehler bei der Messung.
Die Auffächerung des Strahls ist eine weitere 
mögliche Ursache von Messungenauigkeiten bei der Bestimmung der Absorptionskoeffizienten der Elementarwürfel, da der Strahl auch nebenliegende
Elementarwürfel treffen kann und so die gemessene Intensität beeinflusst.
Des Weiteren ist die Justierung der Würfel im Allgemeinen eine Fehlerquelle die sich auf die Messung auswirken kann.