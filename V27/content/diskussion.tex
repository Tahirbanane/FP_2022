\section{Diskussion}
\label{sec:Diskussion}
Ein systematischer Fehler ist die Ungenauigkeit bei der Abstandsmessung der Interferenzmaxima die auf $x=10$ px geschätzt wurde.
Der experimentell bestimmte Landé-Faktor der roten $\sigma$-Spektrallinie $g_{Rot}^{exp} = 0.94 \pm 0.04$ weicht um 6\% von dem theoretischen Wert von  $g_{Rot}^{theo} = 1$ ab.
Bei der blauen $\sigma$-Spektrallinie ergibt sich der theoretische Wert aus den beiden Faktoren $g_1=1.5$ und $g_2=2$ zu $g_{Blau}^{theo} = 1.75$.
Somit liegt eine Abweichung von 21\% zum experimentell bestimmten Landé-Faktor von $g_{Blau}^{exp} = 1.37 \pm 0.10$.
Da die Auflösung der Lummer-Gehrcke-Platte zu gering ist um die kleine Aufspaltung sichtbar zu machen, wird der Mittelwert der Landé-Faktoren als Theoriewert angenommen.

\noindent
Diese Abweichungen können unter anderem durch die nicht exakte Bestimmung des verwendeten Magnetfelds hervorgerufen werden.
Mit 

\begin{equation*}
    \label{eqn:bid}
    B=\frac{hc}{4\lambda^4}\frac{\lambda_D}{\mu_B g}
\end{equation*}

\noindent
kann die ideale Magnetfeldstärke für die Beobachtung der Aufspaltung der verschiedenen Spektrallinien bestimmt werden.
Für die rote $\sigma$-Linie liegt diese bei $B=632$ mT, die selbst bei einer maximalen Stromstärke von $I=5$ A nicht erreicht werden konnte.
Die Interferenzmuster der blauen $\pi$-Linie sind in Abbildung \ref{fig:blauu} dargestellt.
Da in Abbildung \ref{fig:blau2b} keine Aufspaltung zu erkennen ist, ist eine Auswertung aufgrund der unzureichenden Auflösung nicht sinnvoll.
Ein Grund dafür könnte die zu hohe optimale Magnetfeldstärke sein, denn diese beträgt nach Gleichung \ref{eqn:bid} 1.253 T.
Die Magnetfeldstärke betrug bei der Durchführung $316.16 \pm 24.72$ mT und somit wurde die ideale Magnetfeldstärke bei weitem nicht erreicht.


\begin{figure}[H]
	\centering
	\begin{subfigure}[b]{0.8\textwidth}
		\centering
		\includegraphics[width=0.8\textwidth]{data/blau2.JPG}
		\caption{Ohne Magnetfeld.}
    \label{fig:blau2}
	\end{subfigure}
	
	\begin{subfigure}[b]{0.8\textwidth}
		\centering
		\includegraphics[width=0.8\textwidth]{data/blau2b.JPG}
		\caption{Magnetfeld von $B= 316.16$ mT.}
    \label{fig:blau2b}
	\end{subfigure}
    \caption{Interferenzmuster der blauen Spektrallinie einer Cd-Lampe.}
\label{fig:blauu}
\end{figure}


