\section{Zielsetzung}
Ziel des Versuches ist es den Zeeman-Effekt anhand von einer Aufspaltung von Spektralinien von Cadmium und die dabei entstehende Polarisation zu untersuchen.

\section{Magnetisches Moment eines Elektrons}
Allgemein können gebundenen Elektronen ein Bahndrehimpuls $\vec{l}$ und Spin $\vec{s}$ zugeordnet werden. Da die Summen der Drehimpulse in inneren Schalen sich auf 0 addiert, werden
im folgenden nur die äußeren Schalen betrachtet. 

\noindent
Aus den Eigenwertgleichungen der Atome werden die Beträge der Quantenzahlen $l$ und $s$ bestimmt zu 

\vspace{-25pt}
\begin{align}
    |\vec{l}| &= \hbar \sqrt{l(l+1)} & |\vec{s}| &= \hbar \sqrt{s(s+1)} \, .
\end{align}

\noindent
Der Drehimpuls kann ganzezahlige Werte von 0 bis n-1 annehmen, wobei n die Hauptquantenzahl darstellt. Der Spin hat den Wert $s=\frac{1}{2}$. 

\noindent
Beiden Quantenzahlen kann ein magnetisches Moment zugeordnet werden, was klassisch gesprochen, durch die Bewegungen des Elektrons entsteht. 

\vspace{-15pt}
\begin{align}
    \vec{\mu_l} &= - \frac{\mu_\text{B}}{\hbar} \cdot \vec{l}, & |\vec{\mu_l}|& = - \mu_\text{B} \sqrt{l(l+1)}\\
    \vec{\mu_s} &= - g_s \frac{\mu_\text{B}}{\hbar} \cdot \vec{s}, & |\vec{\mu_s}| &= - g_s \mu_\text{B} \sqrt{s(s+1)} \, ,
\end{align}

\noindent
wobei $\mu_B = -\frac{e_0 \hbar}{2 m_0}$ dem Bohrschen Magneton enspricht. 

\section{Wechselwirkung Drehimpuls}







