\section{Zielsetzung}
Ziel des Versuches ist es den Zeeman-Effekt anhand von einer Aufspaltung von Spektralinien von Cadmium und die dabei entstehende Polarisation zu untersuchen.

\section{Magnetisches Moment eines Elektrons}
Allgemein können gebundenen Elektronen ein Bahndrehimpuls $\vec{l}$ und Spin $\vec{s}$ zugeordnet werden. Da die Summen der Drehimpulse in inneren Schalen sich auf 0 addiert, werden
im folgenden nur die äußeren Schalen betrachtet. 

\noindent
Aus den Eigenwertgleichungen der Atome werden die Beträge der Quantenzahlen $l$ und $s$ bestimmt zu 

\vspace{-25pt}
\begin{align}
    |\vec{l}| &= \hbar \sqrt{l(l+1)} & |\vec{s}| &= \hbar \sqrt{s(s+1)} \, .
\end{align}

\noindent
Der Drehimpuls kann ganzezahlige Werte von 0 bis n-1 annehmen, wobei n die Hauptquantenzahl darstellt. Der Spin hat den Wert $s=\frac{1}{2}$. 

\noindent
Beiden Quantenzahlen kann ein magnetisches Moment zugeordnet werden, was klassisch gesprochen, durch die Bewegungen des Elektrons entsteht. 

\vspace{-15pt}
\begin{align}
    \vec{\mu_l} &= - \frac{\mu_\text{B}}{\hbar} \cdot \vec{l}, & |\vec{\mu_l}|& = - \mu_\text{B} \sqrt{l(l+1)}\\
    \vec{\mu_s} &= - g_s \frac{\mu_\text{B}}{\hbar} \cdot \vec{s}, & |\vec{\mu_s}| &= - g_s \mu_\text{B} \sqrt{s(s+1)} \, ,
\end{align}

\noindent
wobei $\mu_B = -\frac{e_0 \hbar}{2 m_0}$ dem Bohrschen Magneton enspricht. 

\section{Wechselwirkung Drehimpuls}
Im allgemeinen kann der Spin und der Drehimpuls eines Teilchens mit einander wechselwirken. Bei gebundenen Teilchen führt 
diese Wechselwirkung, auch Spin-Bahn-Kopplung oder Spin-Bahn-Wechselwirkung genannt, zu einer Aufspaltung von Energieniveaus.

\subsection{LS-Kopplung}
Bei Atomen kann die Spin-Bahn-Wechselwirkung mit der LS-Kopplung genähert werden. Das liegt daran, dass L und S näherungsweise mit dem Hamiltonoperator des Systems
kommutieren oder anders gesagt, die einzelnen Elektronen sind von anderen Elektronen nicht besonders stark abgeschirmt, sodass diese den Bahndrehimpuls
und den Spin von einander "spüren". Dabei sind L und S definiert als 

\begin{align}
    \qquad \vec{L} &= \sum_i \vec{l_i} \:, & |\vec{L}| = \hbar \sqrt{L(L+1)}\\
    \qquad \vec{S} &= \sum_i \vec{s_i} \:, & |\vec{S}| = \hbar \sqrt{S(S+1)} \, , 
\end{align}
\vspace{-10pt}

\noindent
wobei $\vec{l_i}$ den Bahndrehimpuls eines einzelnen Elektrons und $\vec{s_i}$ den Spin eines einzelnen Elektrons beschreibt. (Für den Nahndrehimpuls reicht es lediglich
die äußerste Schale zu betrachten, da sich der gesamt Drehimpuls einer vollen Schale aufhebt.)

\noindent
Somit kann der Gesamtdrehimpuls der Elektronenhülle bestimmt werden zu 

\vspace{-15pt}
\begin{align}
    \vec{J} &= \vec{L} + \vec{S} \:, & |\vec{J}| = \hbar \sqrt{J(J+1)} \: .
\end{align}


\noindent
L und S kann können zusätzlich jeweils magnetische Momente zugeordnet werden. 

\vspace{-15pt}
\begin{align}
 |\vec{\mu_L}|& = \mu_\text{B} \sqrt{L(L+1)} \:, &  |\vec{\mu_S}|& = g_S \mu_\text{B} \sqrt{S(S+1)}\: .
\end{align}

\noindent
Aus diesen kann Analog zum Gesamtdrehimpuls ein magnetisches Moment berechnet werden

\vspace{-15pt}
\begin{align}
    \vec{\mu_J} &= \vec{\mu_L} + \vec{\mu_S} \:, & |\vec{\mu_J}| = g_J \mu_\text{B} \sqrt{J(J+1)} \, .
\end{align}

\noindent
Wichtig bei diesem magnetischen Moment ist das die Richtungen von $\vec{J}$ und $\vec{\mu_J}$ meist nicht übereinstimmt, sodass
nur die zu $\mu_J$ parallele $\vec{J}$-Komponente berücksichtigt wird. Der Landé-Faktor beträgt 

\begin{equation}
    g_J = \frac{3J(J+1) + S(S+1) - L(L+1)}{2J(J+1)}\, .
    \label{eqn:lande}
\end{equation}


\subsubsection{jj-Kopplung}

Bei Kernen mit einer hohen Ordnungszahl sind die einzelnen Elektronen stärker von einander abgeschirmt. Dies für dazu, dass diese ihren eigenen Drehimpuls und Spin stärker 
spüren als den der anderen Elektronen. 

\noindent
Dies führt dazu, dass Gesamtdrehimpulse $\vec{J}$ der Elektronenhülle aus den einzelnen Gesamtdrehimpulsen der Elektronen berechnet wird, sodass gilt
\begin{equation}
    \vec{J} = \sum_i \vec{j_i} \; .
\end{equation}


% Auswahlregeln normaler Zeeman, anormaler zeeman